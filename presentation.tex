\documentclass{beamer}

% american mathematical society
\usepackage{amsmath,amsthm,amssymb}

% additional mathematical fonts
\usepackage{mathrsfs}

% remove navigation bar
\beamertemplatenavigationsymbolsempty

% add slide numbers
\setbeamertemplate{footline}[frame number]

% title
\title{Financial Engineering}
\subtitle{Exam}
\author{Kasper Rosenkrands}
\institute{Aalborg University}
\date{F20}

% toc at each section
\AtBeginSection[]
{
  \begin{frame}
    \frametitle{Table of Contents}
    \tableofcontents[currentsection, hideothersubsections]
  \end{frame}
}

\begin{document}

\frame{\titlepage}

\begin{frame}
\frametitle{Table of Contents}
\tableofcontents[hideallsubsections]
\end{frame}

\section{Bonds and the Value of Money Through Time}

\subsection{Bonds}

\begin{frame}\frametitle{{\normalsize \secname} \\ {\large \subsecname}}
    \begin{definition}[Bond]\label{def:bond}
        A bond is a financial security that pays the owner a chain of predetermined payments.
    \end{definition}
    \pause
    \begin{itemize}
        \item Financial asset with no risk
    \end{itemize}
\end{frame}

%\subsection{Value of Money Through Time}
%
%\begin{frame}\frametitle{{\normalsize \secname} \\ {\large \subsecname}}
%    \begin{itemize}
%        \item You are offered an amount of money with value $M$
%        \begin{itemize}
%            \item Recieve right away
%            \item Recieve one year from now
%        \end{itemize}
%    \end{itemize}
%\end{frame}

\subsection{Interest Rate}

\begin{frame}\frametitle{{\normalsize \secname} \\ {\large \subsecname}}
    \begin{itemize}
        \item Predetermined payments are also known as interest
        \item Fraction of an investment paid either ones for several periods
        \item Different types of interest
        \begin{enumerate}
            \item Simple
            \item Compounded
            \item Continuously compounded
        \end{enumerate}
    \end{itemize}
\end{frame}

\begin{frame}\frametitle{{\normalsize \secname} \\ {\large \subsecname}}
    \begin{definition}[Wealth Process]
        The evolution of an investment over time is called the wealth process of that investment and is denoted by
        \begin{align}
            V = (V_t)_{0 \leq t \leq T}.
        \end{align}
        The initial capital is denoted by $v_0$, and we assume that $V$ is a real-valued stochastic process on a given probability space $(\Omega, \mathscr{F}, \mathbb{P})$.
    \end{definition}
\end{frame}

\begin{frame}\frametitle{{\normalsize \secname} \\ {\large \subsecname}}
    \begin{definition}[Simple Interest]
        Let $v_0 \in \mathbb{R}$ be our initial capital.
        An interest on $v_0$ is said to be simple if it follows the wealth process
        \begin{align}\label{eq:simple_interest_wealth_process}
            V_t = (1 + rt)v_0, \quad 0 \leq t \leq T.
        \end{align}
    \end{definition}
\end{frame}

\begin{frame}\frametitle{{\normalsize \secname} \\ {\large \subsecname}}
    I will now show that the wealth process in \eqref{eq:simple_interest_wealth_process} is indeed a stochastic process in any probability space.
    Any stochastic process $X$ on the probability space $(\Omega, \mathscr{F}, \mathbb{P})$ satisfies
    \begin{align}
        \{\omega \in \Omega \, : \, X(\omega) \leq x\} \in \mathscr{F}, \quad \forall x \in \mathbb{R}.
    \end{align}
    Suppose $v_0 > 0$ and $x \geq (1 + rt)v_0$ then
    \begin{align}
        \{\omega \in \Omega \, : \, (1 + rt)v_0 \leq x\} = \{\Omega\} \in \mathscr{F},
    \end{align}
    on the other hand if $x < (1 + rt)v_0$
    \begin{align}
        \{\omega \in \Omega \, : \, (1 + rt)v_0 \leq x\} = \{\emptyset\} \in \mathscr{F}.
    \end{align}
    As both $\Omega$ and $\emptyset$ is contained in any $\sigma$-algebra we have shown that the wealth process in \eqref{eq:simple_interest_wealth_process} is a stochastic process in any probability space.
\end{frame}

\begin{frame}\frametitle{{\normalsize \secname} \\ {\large \subsecname}}
    \begin{definition}[Compounded Interest]
        Let $v_0 \in \mathbb{R}$ be our initial capital.
        An interest on $v_0$ is said to be compunded over $m\in\mathbb{N}$ periods if it follows the wealth process
        \begin{align}\label{eq:compund_interest_wealth_process}
            V_t = \left( 1 + \frac{r}{m} \right)^{mt} v_0, \quad 0 \leq t \leq T.
        \end{align}
    \end{definition}
\end{frame}

\begin{frame}\frametitle{{\normalsize \secname} \\ {\large \subsecname}}
    Note that we have the following properties $\forall\, 0 \leq t \leq T$
    \begin{enumerate}
        \item $V_{t + 1} = \left( 1 + \dfrac{r}{m} \right)^{m}V_t$,
        \item If $m_1 > m_2, v_0 > 0 
        \, \Rightarrow \, 
        \left( 1 + \dfrac{r}{m_1} \right)^{m_1t}v_0
        >
        \left( 1 + \dfrac{r}{m_2} \right)^{m_2t}v_0$,
        \item If $m_1 > m_2, v_0 < 0 
        \, \Rightarrow \, 
        \left( 1 + \dfrac{r}{m_1} \right)^{m_1t}v_0
        <
        \left( 1 + \dfrac{r}{m_2} \right)^{m_2t}v_0$.
    \end{enumerate}
    From this is follows that for an \textit{investor} compund interest is more attractive as it pays more, however as a \textit{debtor} it is less attractive as he or she will have to pay more on his or hers debt.
\end{frame}

\begin{frame}\frametitle{{\normalsize \secname} \\ {\large \subsecname}}
    At last i can turn to continuously compounded interest which i will present as the limit of \eqref{eq:compund_interest_wealth_process} as $m \rightarrow \infty$.
    Note that by the following definition of $e$
    \begin{align}
        \lim_{x \rightarrow \infty} \left(1 + \frac{1}{x}\right)^x = e,
    \end{align}
    by letting $x = r/m$ in the above the limit of the wealth process of compounded interest can be seen as
    \begin{align}
        \left[\left( 1 + \frac{r}{m} \right)^\frac{m}{r}\right]^{rt}v_0 \rightarrow
        (e)^{rt}v_0, \quad \text{as } m \rightarrow \infty.
    \end{align} 
    This leads to the definition of continuously compounded interest.
\end{frame}

\begin{frame}\frametitle{{\normalsize \secname} \\ {\large \subsecname}}
    \begin{definition}[Continuously Compounded Interest]
        Let $v_0$ be our initial capital.
        An interest on $v_0$ is said to be continuously compounded at rate $r>0$ if the wealth process
        \begin{align}
            V_t = e^{rt}v_0, \quad 0 \leq t \leq T.
        \end{align}
    \end{definition}
\end{frame}

\begin{frame}\frametitle{{\normalsize \secname} \\ {\large \subsecname}}
    There exists the following relation between the different types of interest
    \begin{align}
        (1 + r) \leq \left(1 + \frac{r}{m}\right)^m < e^r.
    \end{align}
    To show that the relation indeed holds i will show that the sequence
    \begin{align}
        a_m = \left(1 + \frac{r}{m}\right)^m,
    \end{align}
    is increasing.
\end{frame}

\begin{frame}\frametitle{{\normalsize \secname} \\ {\large \subsecname}}
    Using the binomial theorem
    \begin{align*}
        (x+y)^n = \sum_{k=0}^n \begin{pmatrix} n \\ k \end{pmatrix} x^k y^{n-k}
    \end{align*}
    we have 
    \begin{align*}
        \left( 1 + \frac{r}{m} \right)^m &= \sum_{k=0}^m \begin{pmatrix} m \\ k \end{pmatrix} 1^{m-k}\begin{pmatrix} \frac{r}{m} \end{pmatrix}^k\\
        &= \sum_{k=0}^m \begin{pmatrix} m \\ k \end{pmatrix} \begin{pmatrix} \frac{r}{m} \end{pmatrix}^k := \clubsuit
    \end{align*}
\end{frame}

\begin{frame}\frametitle{{\normalsize \secname} \\ {\large \subsecname}}
    Each term is of the form
    \begin{align*}
        \begin{pmatrix} m \\ k \end{pmatrix}\begin{pmatrix} \frac{r}{m} \end{pmatrix}^k = \prod_{l=0}^{k-1} \frac{m-l}{k - l} \begin{pmatrix} \frac{r}{m} \end{pmatrix}
    \end{align*}
\end{frame}

\begin{frame}\frametitle{{\normalsize \secname} \\ {\large \subsecname}}
    Each term is of the form
    \begin{align*}
        \frac{m-l}{k-l} \frac{r}{m} &= \frac{rm - lr}{m(k-l)}\\
        &= \frac{m(r - lr/m)}{m(k - l)}\\
        &= \frac{r - lr/m}{k-l} := \bigstar
    \end{align*}
    The term $\bigstar$ increases with $m$ and thus the product increases with $m$ and thus the sum $\clubsuit$ increases with $m$ and therefore it is an increasing sequence.
\end{frame}

\subsection{Types of Bonds}

\begin{frame}\frametitle{{\normalsize \secname} \\ {\large \subsecname}}
    I will discuss the following two types here
    \begin{enumerate}
        \item zero-coupon bonds,
        \item coupon bonds.
    \end{enumerate}
\end{frame}

\begin{frame}\frametitle{{\normalsize \secname} \\ {\large \subsecname}}
    A \textbf{zero-coupon bond} is a bond with a single payment $F > 0$ at time $T > 0$.
    The pay-off F is called the face value and $T$ the maturity time.
    The next question i will answer is how much i will be willing to pay for such a financial assest.
    This depends on the way the time value of money is measured.
    Consider for example the following setup; let $B_0 \geq 0$ be the value of the zero-coupon bond with face value $F>0$ and maturity time $T>0$.
    Suppose that only annual compound interest at rate $r>0$ is available.
\end{frame}

\begin{frame}\frametitle{{\normalsize \secname} \\ {\large \subsecname}}
    From a buyers perspective what if
    \begin{align}\label{eq:zero_coupon_bond_example}
        B_0 > \frac{F}{(1 + r)^T},
    \end{align}
    would i buy the bond?
    \pause
    
    Suppose now that we flip the inequality and look from a sellers perspective, that is if
    \begin{align}\label{eq:zero_coupon_bond_example_seller}
        B_0 < \frac{F}{(1 + r)^T},
    \end{align}
    would i sell the bond?
\end{frame}

\begin{frame}\frametitle{{\normalsize \secname} \\ {\large \subsecname}}
    Now i will consider the situatuion where at time $1 \leq t \leq T$ i want to get rid of a bond, but i what to determine what price i should sell it to.
    At this time the bond can be considered a new zero-coupon bond with face value $F > 0$ and maturity time $T - t$.
    Thus we have from the previous argumentation that
    \begin{align}
        B_t = \frac{F}{(1 + r)^{T - t}}, \quad 0 \leq t \leq T.
    \end{align} 
\end{frame}

\begin{frame}\frametitle{{\normalsize \secname} \\ {\large \subsecname}}
    The chain of arguments holds also when the time value of money is different, if a compounded interest over $m$ periods where considered then the fair price of a zero-coupon bond at time $t$ would be
    \begin{align}
        B_t = \frac{F}{\left(1 + \frac{r}{m}\right)^{m(T - t)}}.
    \end{align}
    If we consider the continuously compounded case the fair price would be
    \begin{align}
        B_t = \frac{F}{e^{r(T -t)}}.
    \end{align}
\end{frame}

\begin{frame}\frametitle{{\normalsize \secname} \\ {\large \subsecname}}
    how much money will i have to deposit in my bank account today if i want to
    \begin{align}
        \text{1.} \quad &\text{withdraw $C > 0$ after 1 year} \\
        \text{2.} \quad &\text{withdraw $C > 0$ after 2 years} \\
        &\quad \vdots \nonumber \\
        \text{$T-1$.} \quad &\text{withdraw $C > 0$ after $T-1$ years} \\
        \text{$T$.} \quad &\text{withdraw $F + C$ after $T$ years}
    \end{align}
    and have nothing left in the bank account afterwards.
\end{frame}

\begin{frame}\frametitle{{\normalsize \secname} \\ {\large \subsecname}}
    In order to be able to get $C > 0$ after one year i have to put
    \begin{align}
        \frac{C}{1+r}
    \end{align}
    in the bank.
    \pause

    In order to be able to get $C > 0$ after two years i have to put 
    \begin{align}
        \frac{C}{(1+r)^2}
    \end{align}
    in the bank.
\end{frame}

\begin{frame}\frametitle{{\normalsize \secname} \\ {\large \subsecname}}
    Generalizing this argument tells me that in order to recieve $C > 0$ after $t$ years i have to put
    \begin{align}
        \frac{C}{(1+r)^t}
    \end{align}
    in the bank.
    \pause

    Lastly in order to get $F + C$ after $T$ years i have to put
    \begin{align}
        \frac{F + C}{(1+r)^T} = \frac{F}{(1+r)^T} + \frac{C}{(1+r)^T}
    \end{align}
    in the bank.
\end{frame}

\begin{frame}\frametitle{{\normalsize \secname} \\ {\large \subsecname}}
    Adding up all these amounts it is concluded that i have to make a deposit of
    \begin{align}
        \sum_{i = 1}^T \frac{C}{(1 + r)^i} + \frac{F}{(1+r)^T}.
    \end{align}\pause
    The agreeable price of a coupon bond is thus given by
    \begin{align}
        B_0 = 
        \sum_{i = 1}^T \frac{C}{(1 + r)^i} + \frac{F}{(1+r)^T} =
        \frac{\xi_T}{(1 + r)^T}.
    \end{align}\pause
    Where the pay-off $\xi_T$ at time $T$ is given by
    \begin{align}
        \xi_T := \sum_{i = 1}^T C(1 + r)^{T -i} + F,
    \end{align}\pause
    in other words the fair price of the coupon bond (as well as the zero-coupon bond) can be written as the discounted price of the total pay-off.
\end{frame}

\section{Portfolio Allocation and Risk Measures}
\subsection{Portfolio}
\subsection{Risk Measures}
\subsection{Coherent Risk Measures}

\section{The Multi-Step Binomial Model}
\subsection{Model Setup}
\subsection{Market Information}
\subsection{Absence of Arbitrage}
\subsection{Risk-Neutral Measure}

\section{The First Fundamental Theorem of Asset Pricing}
\subsection{The Theorem}
\subsection{Explaining the Hypothesis}
\subsection{Risk-Neutral and Martingale Measures}

\section{Pricing in the Binomial Model}
\subsection{Methodology for Pricing}
\subsection{Define a model for the prices}
\subsection{Indicate a set of information}
\subsection{Check for arbitrage opportunities}
\subsection{Dynamics in the risk-neutral world}
\subsection{Pricing a Call Option}
\subsection{Price function for simple derivatives}
\subsection{General Price of a Call option in the Binomial Model}

\section{The Second Fundamental Theorem of Asset Pricing}
\subsection{Market}
\subsection{The Theorem}
\subsection{Definitions}
\subsection{Remarks}

\end{document}