\documentclass{beamer}

\setbeamertemplate{section in toc}[sections numbered]
\setbeamertemplate{subsection in toc}[subsections numbered]

% american mathematical society
\usepackage{amsmath,amsthm,amssymb}

% equation numbering
\numberwithin{equation}{section}

% theorem types
\newtheorem{proposition}{Proposition}

% math operators
\DeclareMathOperator*{\argmin}{argmin}
\DeclareMathOperator*{\var}{Var}
\DeclareMathOperator*{\cov}{Cov}
\DeclareMathOperator{\VaR}{VaR}
\DeclareMathOperator{\cvar}{CVaR}

% additional mathematical fonts
\usepackage{mathrsfs}

% remove navigation bar
\beamertemplatenavigationsymbolsempty

% add slide numbers
\setbeamertemplate{footline}[frame number]

% title
\title{Financial Engineering}
\subtitle{Exam}
\author{Kasper Rosenkrands}
\institute{Aalborg University}
\date{F20}

% toc at each section
\AtBeginSection[]
{
  \begin{frame}
    \frametitle{Table of Contents}
    \tableofcontents[currentsection, hideothersubsections]
  \end{frame}
}


% definition of comment function itself
\newcommand{\comment}[1]{
    \begin{center}
        \colorbox{yellow}{
            \textsf{
                \textbf{#1}
            }
        }
    \end{center}
}
\newcommand{\task}[1]{
    \begin{center}
        \colorbox{red}{
            \textsf{
                \textbf{#1}
            }
        }
    \end{center}
}

\begin{document}

\frame{\titlepage}

\begin{frame}
\frametitle{Table of Contents}
\tableofcontents[hideallsubsections]
\end{frame}

\section{Bonds and the Value of Money Through Time}

\subsection{Bonds}

\begin{frame}\frametitle{{\normalsize \secname} \\ {\large \subsecname}}
    \begin{definition}[Bond]\label{def:bond}
        A bond is a financial security that pays the owner a chain of predetermined payments.
    \end{definition}
    \pause
    \begin{itemize}
        \item Financial asset with no risk
    \end{itemize}
\end{frame}

%\subsection{Value of Money Through Time}
%
%\begin{frame}\frametitle{{\normalsize \secname} \\ {\large \subsecname}}
%    \begin{itemize}
%        \item You are offered an amount of money with value $M$
%        \begin{itemize}
%            \item Recieve right away
%            \item Recieve one year from now
%        \end{itemize}
%    \end{itemize}
%\end{frame}

\subsection{Interest Rate}

\begin{frame}\frametitle{{\normalsize \secname} \\ {\large \subsecname}}
    \begin{itemize}
        \item Predetermined payments are also known as interest
        \item Fraction of an investment paid either ones for several periods
        \item Different types of interest
        \begin{enumerate}
            \item Simple
            \item Compounded
            \item Continuously compounded
        \end{enumerate}
    \end{itemize}
\end{frame}

\begin{frame}\frametitle{{\normalsize \secname} \\ {\large \subsecname}}
    \begin{definition}[Wealth Process]
        The evolution of an investment over time is called the wealth process of that investment and is denoted by
        \begin{align}
            V = (V_t)_{0 \leq t \leq T}.
        \end{align}
        The initial capital is denoted by $v_0$, and we assume that $V$ is a real-valued stochastic process on a given probability space $(\Omega, \mathscr{F}, \mathbb{P})$.
    \end{definition}
\end{frame}

\begin{frame}\frametitle{{\normalsize \secname} \\ {\large \subsecname}}
    \begin{definition}[Simple Interest]
        Let $v_0 \in \mathbb{R}$ be our initial capital.
        An interest on $v_0$ is said to be simple if it follows the wealth process
        \begin{align}\label{eq:simple_interest_wealth_process}
            V_t = (1 + rt)v_0, \quad 0 \leq t \leq T.
        \end{align}
    \end{definition}
\end{frame}

\begin{frame}\frametitle{{\normalsize \secname} \\ {\large \subsecname}}
    I will now show that the wealth process in \eqref{eq:simple_interest_wealth_process} is indeed a stochastic process in any probability space.
    Any stochastic process $X$ on the probability space $(\Omega, \mathscr{F}, \mathbb{P})$ satisfies
    \begin{align}
        \{\omega \in \Omega \, : \, X(\omega) \leq x\} \in \mathscr{F}, \quad \forall x \in \mathbb{R}.
    \end{align}
    Suppose $v_0 > 0$ and $x \geq (1 + rt)v_0$ then
    \begin{align}
        \{\omega \in \Omega \, : \, (1 + rt)v_0 \leq x\} = \{\Omega\} \in \mathscr{F},
    \end{align}
    on the other hand if $x < (1 + rt)v_0$
    \begin{align}
        \{\omega \in \Omega \, : \, (1 + rt)v_0 \leq x\} = \{\emptyset\} \in \mathscr{F}.
    \end{align}
    As both $\Omega$ and $\emptyset$ is contained in any $\sigma$-algebra we have shown that the wealth process in \eqref{eq:simple_interest_wealth_process} is a stochastic process in any probability space.
\end{frame}

\begin{frame}\frametitle{{\normalsize \secname} \\ {\large \subsecname}}
    \begin{definition}[Compounded Interest]
        Let $v_0 \in \mathbb{R}$ be our initial capital.
        An interest on $v_0$ is said to be compunded over $m\in\mathbb{N}$ periods if it follows the wealth process
        \begin{align}\label{eq:compund_interest_wealth_process}
            V_t = \left( 1 + \frac{r}{m} \right)^{mt} v_0, \quad 0 \leq t \leq T.
        \end{align}
    \end{definition}
\end{frame}

\begin{frame}\frametitle{{\normalsize \secname} \\ {\large \subsecname}}
    Note that we have the following properties $\forall\, 0 \leq t \leq T$
    \begin{enumerate}
        \item $V_{t + 1} = \left( 1 + \dfrac{r}{m} \right)^{m}V_t$,
        \item If $m_1 > m_2, v_0 > 0 
        \, \Rightarrow \, 
        \left( 1 + \dfrac{r}{m_1} \right)^{m_1t}v_0
        >
        \left( 1 + \dfrac{r}{m_2} \right)^{m_2t}v_0$,
        \item If $m_1 > m_2, v_0 < 0 
        \, \Rightarrow \, 
        \left( 1 + \dfrac{r}{m_1} \right)^{m_1t}v_0
        <
        \left( 1 + \dfrac{r}{m_2} \right)^{m_2t}v_0$.
    \end{enumerate}
    From this is follows that for an \textit{investor} compund interest is more attractive as it pays more, however as a \textit{debtor} it is less attractive as he or she will have to pay more on his or hers debt.
\end{frame}

\begin{frame}\frametitle{{\normalsize \secname} \\ {\large \subsecname}}
    At last i can turn to continuously compounded interest which i will present as the limit of \eqref{eq:compund_interest_wealth_process} as $m \rightarrow \infty$.
    Note that by the following definition of $e$
    \begin{align}
        \lim_{x \rightarrow \infty} \left(1 + \frac{1}{x}\right)^x = e,
    \end{align}
    by letting $x = r/m$ in the above the limit of the wealth process of compounded interest can be seen as
    \begin{align}
        \left[\left( 1 + \frac{r}{m} \right)^\frac{m}{r}\right]^{rt}v_0 \rightarrow
        (e)^{rt}v_0, \quad \text{as } m \rightarrow \infty.
    \end{align} 
    This leads to the definition of continuously compounded interest.
\end{frame}

\begin{frame}\frametitle{{\normalsize \secname} \\ {\large \subsecname}}
    \begin{definition}[Continuously Compounded Interest]
        Let $v_0$ be our initial capital.
        An interest on $v_0$ is said to be continuously compounded at rate $r>0$ if the wealth process
        \begin{align}
            V_t = e^{rt}v_0, \quad 0 \leq t \leq T.
        \end{align}
    \end{definition}
\end{frame}

\begin{frame}\frametitle{{\normalsize \secname} \\ {\large \subsecname}}
    There exists the following relation between the different types of interest
    \begin{align}
        (1 + r) \leq \left(1 + \frac{r}{m}\right)^m < e^r.
    \end{align}
    To show that the relation indeed holds i will show that the sequence
    \begin{align}
        a_m = \left(1 + \frac{r}{m}\right)^m,
    \end{align}
    is increasing.
\end{frame}

\begin{frame}\frametitle{{\normalsize \secname} \\ {\large \subsecname}}
    Using the binomial theorem
    \begin{align*}
        (x+y)^n = \sum_{k=0}^n \begin{pmatrix} n \\ k \end{pmatrix} x^k y^{n-k}
    \end{align*}
    we have 
    \begin{align*}
        \left( 1 + \frac{r}{m} \right)^m &= \sum_{k=0}^m \begin{pmatrix} m \\ k \end{pmatrix} 1^{m-k}\begin{pmatrix} \frac{r}{m} \end{pmatrix}^k\\
        &= \sum_{k=0}^m \begin{pmatrix} m \\ k \end{pmatrix} \begin{pmatrix} \frac{r}{m} \end{pmatrix}^k := \clubsuit
    \end{align*}
\end{frame}

\begin{frame}\frametitle{{\normalsize \secname} \\ {\large \subsecname}}
    Each term is of the form
    \begin{align*}
        \begin{pmatrix} m \\ k \end{pmatrix}\begin{pmatrix} \frac{r}{m} \end{pmatrix}^k = \prod_{l=0}^{k-1} \frac{m-l}{k - l} \begin{pmatrix} \frac{r}{m} \end{pmatrix}
    \end{align*}
\end{frame}

\begin{frame}\frametitle{{\normalsize \secname} \\ {\large \subsecname}}
    Each term is of the form
    \begin{align*}
        \frac{m-l}{k-l} \frac{r}{m} &= \frac{rm - lr}{m(k-l)}\\
        &= \frac{m(r - lr/m)}{m(k - l)}\\
        &= \frac{r - lr/m}{k-l} := \bigstar
    \end{align*}
    The term $\bigstar$ increases with $m$ and thus the product increases with $m$ and thus the sum $\clubsuit$ increases with $m$ and therefore it is an increasing sequence.
\end{frame}

\subsection{Types of Bonds}

\begin{frame}\frametitle{{\normalsize \secname} \\ {\large \subsecname}}
    I will discuss the following two types here
    \begin{enumerate}
        \item zero-coupon bonds,
        \item coupon bonds.
    \end{enumerate}
\end{frame}

\begin{frame}\frametitle{{\normalsize \secname} \\ {\large \subsecname}}
    A \textbf{zero-coupon bond} is a bond with a single payment $F > 0$ at time $T > 0$.
    The pay-off F is called the face value and $T$ the maturity time.
    The next question i will answer is how much i will be willing to pay for such a financial assest.
    This depends on the way the time value of money is measured.
    Consider for example the following setup; let $B_0 \geq 0$ be the value of the zero-coupon bond with face value $F>0$ and maturity time $T>0$.
    Suppose that only annual compound interest at rate $r>0$ is available.
\end{frame}

\begin{frame}\frametitle{{\normalsize \secname} \\ {\large \subsecname}}
    From a buyers perspective what if
    \begin{align}\label{eq:zero_coupon_bond_example}
        B_0 > \frac{F}{(1 + r)^T},
    \end{align}
    would i buy the bond?
    \pause
    
    Suppose now that we flip the inequality and look from a sellers perspective, that is if
    \begin{align}\label{eq:zero_coupon_bond_example_seller}
        B_0 < \frac{F}{(1 + r)^T},
    \end{align}
    would i sell the bond?
\end{frame}

\begin{frame}\frametitle{{\normalsize \secname} \\ {\large \subsecname}}
    Now i will consider the situatuion where at time $1 \leq t \leq T$ i want to get rid of a bond, but i what to determine what price i should sell it to.
    At this time the bond can be considered a new zero-coupon bond with face value $F > 0$ and maturity time $T - t$.
    Thus we have from the previous argumentation that
    \begin{align}
        B_t = \frac{F}{(1 + r)^{T - t}}, \quad 0 \leq t \leq T.
    \end{align} 
\end{frame}

\begin{frame}\frametitle{{\normalsize \secname} \\ {\large \subsecname}}
    The chain of arguments holds also when the time value of money is different, if a compounded interest over $m$ periods where considered then the fair price of a zero-coupon bond at time $t$ would be
    \begin{align}
        B_t = \frac{F}{\left(1 + \frac{r}{m}\right)^{m(T - t)}}.
    \end{align}
    If we consider the continuously compounded case the fair price would be
    \begin{align}
        B_t = \frac{F}{e^{r(T -t)}}.
    \end{align}
\end{frame}

\begin{frame}\frametitle{{\normalsize \secname} \\ {\large \subsecname}}
    how much money will i have to deposit in my bank account today if i want to
    \begin{align}
        \text{1.} \quad &\text{withdraw $C > 0$ after 1 year} \\
        \text{2.} \quad &\text{withdraw $C > 0$ after 2 years} \\
        &\quad \vdots \nonumber \\
        \text{$T-1$.} \quad &\text{withdraw $C > 0$ after $T-1$ years} \\
        \text{$T$.} \quad &\text{withdraw $F + C$ after $T$ years}
    \end{align}
    and have nothing left in the bank account afterwards.
\end{frame}

\begin{frame}\frametitle{{\normalsize \secname} \\ {\large \subsecname}}
    In order to be able to get $C > 0$ after one year i have to put
    \begin{align}
        \frac{C}{1+r}
    \end{align}
    in the bank.
    \pause

    In order to be able to get $C > 0$ after two years i have to put 
    \begin{align}
        \frac{C}{(1+r)^2}
    \end{align}
    in the bank.
\end{frame}

\begin{frame}\frametitle{{\normalsize \secname} \\ {\large \subsecname}}
    Generalizing this argument tells me that in order to recieve $C > 0$ after $t$ years i have to put
    \begin{align}
        \frac{C}{(1+r)^t}
    \end{align}
    in the bank.
    \pause

    Lastly in order to get $F + C$ after $T$ years i have to put
    \begin{align}
        \frac{F + C}{(1+r)^T} = \frac{F}{(1+r)^T} + \frac{C}{(1+r)^T}
    \end{align}
    in the bank.
\end{frame}

\begin{frame}\frametitle{{\normalsize \secname} \\ {\large \subsecname}}
    Adding up all these amounts it is concluded that i have to make a deposit of
    \begin{align}
        \sum_{i = 1}^T \frac{C}{(1 + r)^i} + \frac{F}{(1+r)^T}.
    \end{align}\pause
    The agreeable price of a coupon bond is thus given by
    \begin{align}
        B_0 = 
        \sum_{i = 1}^T \frac{C}{(1 + r)^i} + \frac{F}{(1+r)^T} =
        \frac{\xi_T}{(1 + r)^T}.
    \end{align}\pause
    Where the pay-off $\xi_T$ at time $T$ is given by
    \begin{align}
        \xi_T := \sum_{i = 1}^T C(1 + r)^{T -i} + F,
    \end{align}\pause
    in other words the fair price of the coupon bond (as well as the zero-coupon bond) can be written as the discounted price of the total pay-off.
\end{frame}

\section{Portfolio Allocation and Risk Measures}

\subsection{Portfolio}

\begin{frame}\frametitle{{\normalsize \secname} \\ {\large \subsecname}}
    Let
        \begin{align}
            \mathfrak{M} = \left\{ 
                \left(
                    \Omega, \mathscr{F}, \mathbb{P}
                \right),
                P = \left(
                    B_t, S_t^{(1)}, \ldots, S_t^{(d)}
                \right)_{0 \leq t \leq T}
            \right\},
        \end{align}
        be a finite-horizon financial market.
\end{frame}

\begin{frame}\frametitle{{\normalsize \secname} \\ {\large \subsecname}}
    %First and foremost i will give the definition of a portfolio.
    \begin{definition}[Portfolio and Strategies]
        A portfolio in $\mathfrak{M}$ is a $(d + 1)$-dimensional vector
        \begin{align}
            \Theta_t = \left(\varphi_t, \theta_t^{(1)}, \ldots, \theta_t^{(d)}\right),
        \end{align}
        in which
        \begin{align}
            \Theta_t^j = \text{Number of shares of the $j$'th asset held between time $t-1$ and $t$}.
        \end{align}
        for $j = 1, \ldots, d + 1$.
        The collection $\Theta = \left( \Theta_t \right)_{0 \leq t \leq T}$, with the convention that $\Theta_0 = \Theta_1$, is termed a strategy.
    \end{definition}
\end{frame}

\begin{frame}\frametitle{{\normalsize \secname} \\ {\large \subsecname}}
    For every strategy on market there is an associated wealth process.
    The wealth process for $\Theta$ is defined and denoted by
    \begin{align}
        V_t^{\Theta} = \varphi_t B_t + \sum_{j = 1}^d \theta_t^{(j)} S_t^{(j)} = \Theta_t \cdot P_t, \quad 0 \leq t \leq T.    
    \end{align}
\end{frame}

\subsection{Risk Measures}

\begin{frame}\frametitle{{\normalsize \secname} \\ {\large \subsecname}}
    Any strategy on a given market inherently carries a risk because the return is random, there is in other words no way to predict our profit or losses with certainty.
    There is no one way to measure the risk associated with a strategy, however in the next two sections i will explore two approaches.
    Both of these is based on portfolio allocation as an optimization problem.
\end{frame}

\begin{frame}\frametitle{{\normalsize \secname} \\ {\large \subsecname}}
    The problem is given in this way; solve
    \begin{align}
        \argmin_{\textbf{w}\in\mathbb{R}^{d+1}}
        \mathcal{R}\left( \textbf{w} \cdot \textbf{K}_P \right),
    \end{align}
    subject to:
    \begin{align}
        \sum_{j = 0}^d w_j = 1, \ 
        \mathbb{E}\left[ U(\textbf{w} \cdot \textbf{K}_P) \right] = \mu, \
        \mu \in \mathbb{R},
    \end{align}
    where $\mathcal{R}$ is a measure of risk and $U(\textbf{w} \cdot \textbf{K}_P)$ is the utility of the strategy.
\end{frame}

\begin{frame}\frametitle{{\normalsize \secname} \\ {\large \subsecname}}
    The mean variance approach assumes the uitility function as the identity, that is
    \begin{align}
        U(x) = x.
    \end{align}
    By letting
    \begin{align}
        \mu_\mathbf{K} := \mathbb{E}\left[\mathbf{K}_P\right],
    \end{align}
    it follows that
    \begin{align}
        \mathbb{E}\left[ U(\mathbf{w} \cdot \mathbf{K}_P) \right] = \mathbf{w} \cdot \mu_\mathbf{K}.
    \end{align}
    Thus the optimization problem, in the mean-variance approach becomes
\end{frame}

\begin{frame}\frametitle{{\normalsize \secname} \\ {\large \subsecname}}
    \begin{problem}[Optimization Problem Mean-Variance]
        \begin{align}
            &\quad\argmin_{\mathbf{w}\in\mathbb{R}^{d+1}} \sqrt{\mathbf{w}^\top C \mathbf{w}} \\
            &\text{Subject to:} \nonumber \\
            &\qquad \text{1.) }\sum_{j = 0}^d w_j = 1, \\
            &\qquad \text{2.) }\mathbf{w} \cdot \mu_\mathbf{K} = \mu, \quad \mu \in \mathbb{R}.
        \end{align}
    \end{problem}
\end{frame}

\begin{frame}\frametitle{{\normalsize \secname} \\ {\large \subsecname}}
    An illustrative example to demonstrate why one might consider a different risk measure than the standard deviation is the following.
    Consider two portfolios that generate the following wealth
    \begin{align}
        V_1^{(1)} = 
        \begin{cases}
            1 &\text{with probability $1/2$} \\
            -9 &\text{with probability $1/2$}
        \end{cases}
    \end{align}
    and
    \begin{align}
        V_1^{(2)} = 
        \begin{cases}
            5 &\text{with probability $1/2$} \\
            -5 &\text{with probability $1/2$}
        \end{cases}
    \end{align}
    According to their standard deviation these two portfolios carries the same risk, but one could argue that the first is riskier that the latter.
    The next risk measure i will consider i concerned with controlling losses rather than the variation of the return.
\end{frame}

\begin{frame}\frametitle{{\normalsize \secname} \\ {\large \subsecname}}
    If i denote the outcome of an investment with $V_1$ then my potential losses will be given by $-V_1$.
    Suppose that for some $x \in \mathbb{R}$
    \begin{align}
        -V_1 \leq x.
    \end{align}
    Then to cover my risk of bankruptcy i must keep at least the amount $x$ in my bank account.
    In reality the only thing i can quantify is the chance of that happening, which is denoted by
    \begin{align}
        \mathbb{P}\left( -V_1 \leq x \right).
    \end{align}
\end{frame}

\begin{frame}\frametitle{{\normalsize \secname} \\ {\large \subsecname}}
    This is the motivation behind the risk measure Value at Risk.
    \begin{definition}[Value at Risk]
        Let $0 < \alpha < 1$ and $X$ be a random variable.
        The Value at Risk (VaR) of $X$ is defined and denoted by
        \begin{align}
            \VaR_\alpha(X) := \inf
            \left\{
                x \in \mathbb{R} \, : \,
                \mathbb{P}(X + x \geq 0)
                \geq 1 - \alpha
            \right\}.
        \end{align}
    \end{definition}
    In other words the $\VaR_\alpha(X)$ represents the amount of extra capital i need to hold in order to reduce my risk of bankruptcy to $1 - \alpha$.
\end{frame}

\begin{frame}\frametitle{{\normalsize \secname} \\ {\large \subsecname}}
    An alternative representation of $\VaR$ can be formulated using the fact that
    \begin{align}
        \mathbb{P}(-X \leq x) \geq 1 - \alpha
        \, \Longleftrightarrow \,
        \mathbb{P}(X + x < 0) \leq \alpha,
    \end{align}
    this lets us formulate an equivalent representation given by
    \begin{align}
        \VaR_\alpha(X) := \inf
        \left\{
            x \in \mathbb{R} \, : \,
            \mathbb{P}(X < -x) \leq \alpha
        \right\}.
    \end{align}
\end{frame}

\begin{frame}\frametitle{{\normalsize \secname} \\ {\large \subsecname}}
    \begin{proposition}[Properties of $\VaR$]
        Let $X, Y$ be arbitrary random variables.
        Then, the following holds
        \begin{enumerate}
            \item If $X \geq 0$ almost surely, then $\VaR_\alpha(X) \leq 0$.
            \item For all $y\in \mathbb{R}$ we have that $\VaR_\alpha(X + y) = \VaR_\alpha(X) - y$. In particular $\VaR_\alpha(X + \VaR_\alpha(X)) = 0$.
            \item If $\lambda \geq 0$, then $\VaR_\alpha(\lambda X) = \lambda \VaR_\alpha(X)$.
            \item If $X \geq Y$ almost surely, then $\VaR_\alpha(X) \leq \VaR_\alpha(Y)$.
        \end{enumerate}
    \end{proposition}
\end{frame}

\begin{frame}\frametitle{{\normalsize \secname} \\ {\large \subsecname}}
    Proof:
    \vspace{6cm}
\end{frame}

\subsection{Coherent Risk Measures}

\begin{frame}\frametitle{{\normalsize \secname} \\ {\large \subsecname}}
    %There are one important drawback to measuring risk using $\VaR$ and that is its lack of diversification.
    Note that when using standard deviation as a risk measure we get the following
    \begin{align}
        \sigma(X + Y)^2 = \var(X) + \var(Y) + 2\rho_{X,Y}\sqrt{\var(X)\var(Y)},
    \end{align}
    where $\rho_{X,Y} = \frac{\cov(X,Y)}{\sqrt{\var(X)\var(Y)}} \leq 1$.
    Then
    \begin{align}
        \sigma(X + Y)^2 &= \sigma(X)^2 + \sigma(Y)^2 + 2\rho_{X,Y}\sigma(X)\sigma(Y)\\
        &\leq \sigma(X)^2 + \sigma(Y)^2 + 2\sigma(Y)\sigma(X) = \left[\sigma(X) + \sigma(Y)\right]^2, 
    \end{align}
    which would imply that
    \begin{align}
        \sigma(X + Y) \leq \sigma(X) + \sigma(Y).
    \end{align}
\end{frame}

\begin{frame}\frametitle{{\normalsize \secname} \\ {\large \subsecname}}
    However $\VaR$ as a risk measure is not able to reproduce this, that is in general we do not have that
    \begin{align}
        \VaR_\alpha(X + Y) \leq \VaR_\alpha(X) + \VaR_\alpha(Y).
    \end{align}
    As diversification is a desired i will know introduce the concept of Coherent Risk Measures.
\end{frame}

\begin{frame}\frametitle{{\normalsize \secname} \\ {\large \subsecname}}
    \begin{definition}[Coherent Risk Measure]
        A function $\rho \, : \, L^1 \rightarrow \mathbb{R}$ is said to be a Coherent Risk Measure if
        \begin{enumerate}
            \item If $X \geq 0$ almost surely, then $\rho(X) \leq 0$.
            \item For all $y \in \mathbb{R}$ we have that $\rho(X + y) = \rho(X) - y$.
            \item If $\lambda \geq 0$, then $\rho(\lambda X) = \lambda \rho(X)$.
            \item We have that $\rho(X + Y) \leq \rho(X) + \rho(Y)$.
        \end{enumerate}
    \end{definition}
\end{frame}

\begin{frame}\frametitle{{\normalsize \secname} \\ {\large \subsecname}}
    The Conditional Value at Risk (CVAR) is a common example of a coherent risk measure.

    Given a random variable $X$, we will write
    \begin{align}
        q_\alpha(X) := \inf \{x \in \mathbb{R} \, : \, \mathbb{P}(X \leq x) \geq \alpha\}.
    \end{align}
    With this notation now introduced i can present the definition of CVaR.
    \pause
    \begin{definition}[Condition Value at Risk]
        Let $ 0 < \alpha < 1$ and $X \in L^1$.
        The Conditional Value at Risk or Expected Shortfall of $X$ is defined and denoted buy
        \begin{align}
            \cvar_\alpha := - \frac{1}{\alpha}\int_0^\alpha q_r(X) \, dr.
        \end{align}
    \end{definition}
\end{frame}

\begin{frame}\frametitle{{\normalsize \secname} \\ {\large \subsecname}}
    The name Expected Shortfall comes from the fact that if $X$ has a continuous distribution, then
    \begin{align}
        \cvar_\alpha(X) = - \mathbb{E}[X \, | \, X + \VaR_\alpha(X) \leq 0].
    \end{align}
    Thus, $\cvar_\alpha$ measures the expected losses given that $\VaR_\alpha(X)$ was not enough to cover our position on $X$.
\end{frame}

\begin{frame}\frametitle{{\normalsize \secname} \\ {\large \subsecname}}
\end{frame}

\section{The Multi-Step Binomial Model}
\subsection{Model Setup}
\subsection{Market Information}
\subsection{Absence of Arbitrage}
\subsection{Risk-Neutral Measure}

\section{The First Fundamental Theorem of Asset Pricing}
\subsection{The Theorem}
\subsection{Explaining the Hypothesis}
\subsection{Risk-Neutral and Martingale Measures}

\section{Pricing in the Binomial Model}
\subsection{Methodology for Pricing}
\subsection{Define a model for the prices}
\subsection{Indicate a set of information}
\subsection{Check for arbitrage opportunities}
\subsection{Dynamics in the risk-neutral world}
\subsection{Pricing a Call Option}
\subsection{Price function for simple derivatives}
\subsection{General Price of a Call option in the Binomial Model}

\section{The Second Fundamental Theorem of Asset Pricing}
\subsection{Market}
\subsection{The Theorem}
\subsection{Definitions}
\subsection{Remarks}

\end{document}