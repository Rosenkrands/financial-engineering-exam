\documentclass{beamer}

% title
\title{Financial Engineering}
\subtitle{Exam}
\author{Kasper Rosenkrands}
\institute{Aalborg University}
\date{F20}

% toc at each section
\AtBeginSection[]
{
  \begin{frame}
    \frametitle{Table of Contents}
    \tableofcontents[currentsection, hideothersubsections]
  \end{frame}
}

\begin{document}

\frame{\titlepage}

\begin{frame}
\frametitle{Table of Contents}
\tableofcontents[hideallsubsections]
\end{frame}

\section{Bonds and the Value of Money Through Time}
\subsection{Bonds}

\begin{frame}\frametitle{{\normalsize \secname} \\ {\normalsize \subsecname}}
    Here is some text.
\end{frame}

\subsection{Value of Money Through Time}
\subsection{Interest Rate}
\subsection{Types of Bonds}

\section{Portfolio Allocation and Risk Measures}
\subsection{Portfolio}
\subsection{Risk Measures}
\subsection{Coherent Risk Measures}

\section{The Multi-Step Binomial Model}
\subsection{Model Setup}
\subsection{Market Information}
\subsection{Absence of Arbitrage}
\subsection{Risk-Neutral Measure}

\section{The First Fundamental Theorem of Asset Pricing}
\subsection{The Theorem}
\subsection{Explaining the Hypothesis}
\subsection{Risk-Neutral and Martingale Measures}

\section{Pricing in the Binomial Model}
\subsection{Methodology for Pricing}
\subsection{Define a model for the prices}
\subsection{Indicate a set of information}
\subsection{Check for arbitrage opportunities}
\subsection{Dynamics in the risk-neutral world}
\subsection{Pricing a Call Option}
\subsection{Price function for simple derivatives}
\subsection{General Price of a Call option in the Binomial Model}

\section{The Second Fundamental Theorem of Asset Pricing}
\subsection{Market}
\subsection{The Theorem}
\subsection{Definitions}
\subsection{Remarks}

\end{document}