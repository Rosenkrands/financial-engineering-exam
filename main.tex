\documentclass{article}
\usepackage{amsmath,amsthm,amssymb}
\newtheorem{theorem}{Theorem}[section]
\numberwithin{equation}{section}

\begin{document}

\begin{titlepage}
    \begin{center}
        {\huge\textbf{Financial Engineering}}\\[2mm]
        {\Large Kasper Rosenkrands}\\[2cm]
        {\large MATØK6}\\[2mm]
        {\large Spring 2020}
    \end{center}
    
    \vfill
\end{titlepage}

\section{Bonds and the Value of Money Through Time}
Here you can explain what type of interests we studied and how the fair-price of bonds depends on these types of interest and viceversa.

\newpage

\section{Portfolio Allocation and Risk Measures}
Try to explain what a portfolio is and how we can create (allocate) portfolio based on risk measures. In your discussion about the latter, remember to include the differences/benefits/drawbacks of considering the standard deviation, VaR and CVaR as risk measures.

\newpage

\section{The Multi-Step Binomial Model}
Describe the dynamics of the model. Try to discuss, among other things, the way that the market information is described through time, its absence of arbitrage, and how to find a risk-neutral measure in such model. 

\newpage

\section{The First Fundamental Theorem of Asset Pricing}
Focus on explaining the hypothesis in the theorem (martingale measure, admissible strategies, arbitrage, etc.) and its applications to the pricing of financial derivatives.

\newpage

\section{Pricing in the Binomial Model}
Explain what is the typical methodology for pricing derivatives and how such a methodology works in the Binomial model. Remember to include an example (e.g. the price of call-option).

\newpage

\section{The Second Fundamental Theorem of Asset Pricing}
Focus on explaining the hypothesis in the theorem (martingale measure, admissible strategies, arbitrage, completeness, etc.) and its consequences on the pricing of financial derivatives. You can, in particular, show that the Binomial Model is complete while the Trinomial Model is not.

\end{document}