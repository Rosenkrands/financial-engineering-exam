\documentclass{article}
\usepackage{amsmath,amsthm,amssymb}
\newtheorem{theorem}{Theorem}[section]
\theoremstyle{definition}
\newtheorem{definition}{Definition}[section]
\numberwithin{equation}{section}

\begin{document}

\begin{titlepage}
    \begin{center}
        {\huge\textbf{Financial Engineering}}\\[2mm]
        {\Large Kasper Rosenkrands}\\[2cm]
        {\large MATØK6}\\[2mm]
        {\large Spring 2020}
    \end{center}
    
    \vfill
\end{titlepage}

\section{Bonds and the Value of Money Through Time}
\textit{Here you can explain what type of interests we studied and how the fair-price of bonds depends on these types of interest and viceversa.}

\subsection{Bonds}
I will begin by giving the definition of a bond.
\begin{definition}[Bond]
    A bond is a financial security that pays the owner a chain of predetermined payments.\label{def:bond}
\end{definition}
Using definition \ref{def:bond} we can interpret a bond as a financial asset which have no risk.
In order words we know for a fact how much money we will make on this investment, disregarding the inherent risk of default.

\subsection{Value of Money Through Time}
The value of money changes through time.
Assume you were offered an amount of money with value $M$, furthermore you would have the choice of recieving the money today or 1 year from now.
Choosing to recieve the money now you would be able to spend them straightaway which could be preferable compared to waiting a year.
Prices could also have risen during the waiting time, further diminishing the value.
Last but not least, recieving the money today would enable you to put the money in the bank and thereby recieve a premium.
All these arguments serve the purpose of showing that the value of money changes through time, and that the value $M$ of the money would decrease by waiting a year before recieving the money.

\subsection{Interest Rate}
The premium recieved by depositing money at bank mentioned earlier is described by an interest rate.
This interest rate will in the following be described by a positive number $$r>0.$$
Different types of interest exists and i will describe the following three types
\begin{enumerate}
    \item simple interest,
    \item compouned interest,
    \item continuously compounded interes.
\end{enumerate}
Common for all three types is that they represent a fraction of an investment that is paid either ones or through several periods.

However to further dive into the different types of interest we will first define the notion of a wealth process.
\begin{definition}[Wealth Process]
    The evolution of an investment over time is called the wealth process of that investment and is denoted by
    \begin{align}
        V = (V_t)_{0 \leq t \leq T}.
    \end{align}
    The initial capital is denoted by $v_0$, and we assume that $V$ is a real-valued stochastic process on a given probability space $(\Omega, \mathcal{F}, \mathbb{P})$.
\end{definition}
With this definition in place we are now ready to define the three different types of interest starting with simple interest.
\begin{definition}[Simple Interest]
    Let $v_0 \in \mathbb{R}$ be our initial capital.
    An interest on $v_0$ is said to simple if it follows the wealth process
    \begin{align}\label{eq:simple_interest_wealth_process}
        V_t = (1 + rt)v_0, \quad 0 \leq t \leq T.
    \end{align}
\end{definition}
I will now show that the wealth process in \eqref{eq:simple_interest_wealth_process} is indeed a stochastic process in any probability space.
Any stochastic process $X$ on the probability space $(\Omega, \mathcal{F}, \mathbb{P})$ satisfies
\begin{align}
    \{\omega \in \Omega \, : \, X(\omega) \leq x\} \in \mathcal{F}, \quad \forall x \in \mathbb{R}.
\end{align}
Suppose $v_0 > 0$ and $x \geq (1 + rt)v_0$ then
\begin{align}
    \{\omega \in \Omega \, : \, (1 + rt)v_0 \leq x\} = \{\Omega\} \in \mathcal{F},
\end{align}
on the other hand if $x < (1 + rt)v_0$
\begin{align}
    \{\omega \in \Omega \, : \, (1 + rt)v_0 \leq x\} = \{\emptyset\} \in \mathcal{F}.
\end{align}
As both $\Omega$ and $\emptyset$ is contained in any $\sigma$-algebra we have shown that the wealth process in \eqref{eq:simple_interest_wealth_process} is a stochastic process in any probability space.

\newpage

\section{Portfolio Allocation and Risk Measures}
Try to explain what a portfolio is and how we can create (allocate) portfolio based on risk measures. In your discussion about the latter, remember to include the differences/benefits/drawbacks of considering the standard deviation, VaR and CVaR as risk measures.

\newpage

\section{The Multi-Step Binomial Model}
Describe the dynamics of the model. Try to discuss, among other things, the way that the market information is described through time, its absence of arbitrage, and how to find a risk-neutral measure in such model. 

\newpage

\section{The First Fundamental Theorem of Asset Pricing}
Focus on explaining the hypothesis in the theorem (martingale measure, admissible strategies, arbitrage, etc.) and its applications to the pricing of financial derivatives.

\newpage

\section{Pricing in the Binomial Model}
Explain what is the typical methodology for pricing derivatives and how such a methodology works in the Binomial model. Remember to include an example (e.g. the price of call-option).

\newpage

\section{The Second Fundamental Theorem of Asset Pricing}
Focus on explaining the hypothesis in the theorem (martingale measure, admissible strategies, arbitrage, completeness, etc.) and its consequences on the pricing of financial derivatives. You can, in particular, show that the Binomial Model is complete while the Trinomial Model is not.

\end{document}