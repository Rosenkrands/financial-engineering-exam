\documentclass{article}
\usepackage{amsmath,amsthm,amssymb}
\usepackage{enumitem}
\newtheorem{theorem}{Theorem}[section]
\theoremstyle{definition}
\newtheorem{definition}{Definition}[section]
\numberwithin{equation}{section}

\begin{document}

\begin{titlepage}
    \begin{center}
        {\huge\textbf{Financial Engineering}}\\[2mm]
        {\Large Kasper Rosenkrands}\\[2cm]
        {\large MATØK6}\\[2mm]
        {\large Spring 2020}
    \end{center}
    
    \vfill
\end{titlepage}

\section{Bonds and the Value of Money Through Time}
\textit{Here you can explain what type of interests we studied and how the fair-price of bonds depends on these types of interest and viceversa.}

\subsection{Bonds}
I will begin by giving the definition of a bond.
\begin{definition}[Bond]
    A bond is a financial security that pays the owner a chain of predetermined payments.\label{def:bond}
\end{definition}
Using definition \ref{def:bond} we can interpret a bond as a financial asset which have no risk.
In order words we know for a fact how much money we will make on this investment, disregarding the inherent risk of default.

\subsection{Value of Money Through Time}
The value of money changes through time.
Assume you were offered an amount of money with value $M$, furthermore you would have the choice of recieving the money today or 1 year from now.
Choosing to recieve the money now you would be able to spend them straightaway which could be preferable compared to waiting a year.
Prices could also have risen during the waiting time, further diminishing the value.
Last but not least, recieving the money today would enable you to put the money in the bank and thereby recieve a premium.
All these arguments serve the purpose of showing that the value of money changes through time, and that the value $M$ of the money would decrease by waiting a year before recieving the money.

\subsection{Interest Rate}
The premium recieved by depositing money at bank mentioned earlier is described by an interest rate.
This interest rate will in the following be described by a positive number $$r>0.$$
Different types of interest exists and i will describe the following three types
\begin{enumerate}
    \item simple interest,
    \item compouned interest,
    \item continuously compounded interes.
\end{enumerate}
Common for all three types is that they represent a fraction of an investment that is paid either ones or through several periods.

However to further dive into the different types of interest we will first define the notion of a wealth process.
\begin{definition}[Wealth Process]
    The evolution of an investment over time is called the wealth process of that investment and is denoted by
    \begin{align}
        V = (V_t)_{0 \leq t \leq T}.
    \end{align}
    The initial capital is denoted by $v_0$, and we assume that $V$ is a real-valued stochastic process on a given probability space $(\Omega, \mathcal{F}, \mathbb{P})$.
\end{definition}
With this definition in place we are now ready to define the three different types of interest starting with simple interest.
\begin{definition}[Simple Interest]
    Let $v_0 \in \mathbb{R}$ be our initial capital.
    An interest on $v_0$ is said to be simple if it follows the wealth process
    \begin{align}\label{eq:simple_interest_wealth_process}
        V_t = (1 + rt)v_0, \quad 0 \leq t \leq T.
    \end{align}
\end{definition}
I will now show that the wealth process in \eqref{eq:simple_interest_wealth_process} is indeed a stochastic process in any probability space.
Any stochastic process $X$ on the probability space $(\Omega, \mathcal{F}, \mathbb{P})$ satisfies
\begin{align}
    \{\omega \in \Omega \, : \, X(\omega) \leq x\} \in \mathcal{F}, \quad \forall x \in \mathbb{R}.
\end{align}
Suppose $v_0 > 0$ and $x \geq (1 + rt)v_0$ then
\begin{align}
    \{\omega \in \Omega \, : \, (1 + rt)v_0 \leq x\} = \{\Omega\} \in \mathcal{F},
\end{align}
on the other hand if $x < (1 + rt)v_0$
\begin{align}
    \{\omega \in \Omega \, : \, (1 + rt)v_0 \leq x\} = \{\emptyset\} \in \mathcal{F}.
\end{align}
As both $\Omega$ and $\emptyset$ is contained in any $\sigma$-algebra we have shown that the wealth process in \eqref{eq:simple_interest_wealth_process} is a stochastic process in any probability space.
Next i will define the notion of compund interest.
\begin{definition}[Compounded Interest]
    Let $v_0 \in \mathbb{R}$ be our initial capital.
    An interest on $v_0$ is said to be compunded over $m\in\mathbb{N}$ periods if it follows the wealth process
    \begin{align}\label{eq:compund_interest_wealth_process}
        V_t = \left( 1 + \frac{r}{m} \right)^{mt} v_0, \quad 0 \leq t \leq T.
    \end{align}
\end{definition}
As stated in the definition $m$ can be any natural number, however if $m = 1, 12, 365$ the interest is refered to as annually, monthly and daily compounded, respectively.
Note that we have the following properties $\forall\, 0 \leq t \leq T$
\begin{enumerate}[label = \textit{\roman*})]
    \item $V_{t + 1} = \left( 1 + \dfrac{r}{m} \right)^{m}V_t$,
    \item If $m_1 > m_2, v_0 > 0 
    \, \Rightarrow \, 
    \left( 1 + \dfrac{r}{m_1} \right)^{m_1t}v_0
    >
    \left( 1 + \dfrac{r}{m_2} \right)^{m_2t}v_0$,
    \item If $m_1 > m_2, v_0 < 0 
    \, \Rightarrow \, 
    \left( 1 + \dfrac{r}{m_1} \right)^{m_1t}v_0
    <
    \left( 1 + \dfrac{r}{m_2} \right)^{m_2t}v_0$.
\end{enumerate}
From this is follows that for an \textit{investor} compund interest is more attractive as it pays more, however as a \textit{debtor} it is less attractive as he or she will have to pay more on his or hers debt.

At last i can turn to continuously compounded interest which i will present as the limit of \eqref{eq:compund_interest_wealth_process} as $m \rightarrow \infty$.
Note that by the following definition of $e$
\begin{align}
    \lim_{x \rightarrow \infty} \left(1 + \frac{1}{x}\right)^x = e,
\end{align}
by letting $x = r/m$ in the above the limit of the wealth process of compounded interest can be seen as
\begin{align}
    \left[\left( 1 + \frac{r}{m} \right)^\frac{m}{r}\right]^{rt}v_0 \rightarrow
    (e)^{rt}v_0, \quad \text{as } m \rightarrow \infty.
\end{align} 
This leads to the definition of continuously compounded interest
\begin{definition}[Continuously Compounded Interest]
    Let $v_0$ be our initial capital.
    An interest on $v_0$ is said to be continuously compounded at rate $r>0$ if the wealth process
    \begin{align}
        V_t = e^{rt}v_0, \quad 0 \leq t \leq T.
    \end{align}
\end{definition}
There exists the following relation between the different types of interest
\begin{align}
    (1 + r) \leq \left(1 + \frac{r}{m}\right)^m < e^r.
\end{align}
To show that the relation indeed holds i will show that the sequence
\begin{align}
    a_m = \left(1 + \frac{r}{m}\right)^m,
\end{align}
is increasing.

\subsection{Bonds}
A bond is a financial security that pays to the owner, a chain of predetermined payments.
However there are different types of bonds.
I will discuss the following two types here
\begin{enumerate}
    \item zero-coupon bonds,
    \item coupon bonds.
\end{enumerate}
A \textbf{zero-coupon bond} is a bond with a single payment $F > 0$ at time $T > 0$.
The pay-off F is called the face value and $T$ the maturity time.
The next question i will answer is how much i will be willing to pay for such a financial assest.
This depends on the way the time value of money is measured.
Consider for example the following setup; let $B_0 \geq 0$ be the value of the zero-coupon bond with face value $F>0$ and maturity time $T>0$.
Suppose that only annual compound interest at rate $r>0$ is available.
From a buyers perspective what if
\begin{align}\label{eq:zero_coupon_bond_example}
    B_0 > \frac{F}{(1 + r)^T},
\end{align}
would i buy the bond? 
No, of course not because the right side of \eqref{eq:zero_coupon_bond_example} denote the amount of money i would have to put in the bank today to recieve exactly $F$ at time $T$.
When the aforementioned amount of monwy is less than the price of the bond, i would surely choose to put money in the bank instead of buying the bond, as the bond will pays exactly $F$ at time $T$.
Suppose now that we flip the inequality and look from a sellers perspective, that is if
\begin{align}\label{eq:zero_coupon_bond_example_seller}
    B_0 < \frac{F}{(1 + r)^T},
\end{align}
would i sell the bond? 
No of course not because the right side of \eqref{eq:zero_coupon_bond_example_seller} denotes the amount at i can borrow at time 0 if i agree to pay exactly $F$ at time $T$.
For the bond i would also have to pay $F$ at time zero however i would only recieve $B_0$ at time 0.
Therefore i would not agree to sell the bond in this situation.

This implies that the only price a buyer and seller can agree to is in the situation where
\begin{align}
    B_0 = \frac{F}{(1 + r)^T}.
\end{align}
Now i will consider the situatuion where at time $1 \leq t \leq T$ i want to get rid of a bond, but i what to determine what price i should sell it to.
At this time the bond can be considered a new zero-coupon bond with face value $F > 0$ and maturity time $T - t$.
Thus we have from the previous argumentation that
\begin{align}
    B_t = \frac{F}{(1 + r)^{T - t}}, \quad 0 \leq t \leq T.
\end{align} 

The chain of argumenation holds also when the time value of money is different, if a compounded interest over $m$ periods where considered then the fair price of a zero-coupon bond at time $t$ would be
\begin{align}
    B_t = \frac{F}{\left(1 + \frac{r}{m}\right)^{m(T - t)}}.
\end{align}
If we consider the continuously compounded case the fair price would be
\begin{align}
    B_t = \frac{F}{e^{r(T -t)}}.
\end{align}

A \textbf{coupon bond} guarantees the owner a payment $F > 0$ at time $T > 0$, as well as a fixed amount $C > 0$ at $t = 1,2,\ldots,T$.
To find a fair price for such a bond i will follow the same argumentation as above, namely i will compare the price of the bond with the cost of recieving the same benefits with a bank account.
I other words how much money will i have to deposit in my bank account today if i want to
\begin{align*}
    \text{1.} \quad &\text{withdraw $C > 0$ after 1 year} \\
    \text{2.} \quad &\text{withdraw $C > 0$ after 2 years} \\
    &\quad \vdots \\
    \text{$T-1$.} \quad &\text{withdraw $C > 0$ after $T-1$ years} \\
    \text{$T$.} \quad &\text{withdraw $F + C$ after $T$ years} \\
\end{align*}
and have nothing left in the bank account afterwards.

In order to be able to get $C > 0$ after one year i have to put
\begin{align}
    \frac{C}{1+r}
\end{align}
in the bank.
In order to be able to get $C > 0$ after two years i have to put 
\begin{align}
    \frac{C}{(1+r)^2}
\end{align}
in the bank.
Generalizing this argumentat tells me that in order to $C > 0$ after $t$ years i have to put
\begin{align}
    \frac{C}{(1+r)^t}
\end{align}
in the bank.
Lastly in order to get $F + C$ after $T$ years i have to put
\begin{align}
    \frac{F + C}{(1+r)^T} = \frac{F}{(1+r)^T} + \frac{C}{(1+r)^T}
\end{align}
in the bank.
Adding up all these amounts it is concluded that i have to make a deposit of
\begin{align}
    \sum_{i = 1}^T \frac{C}{(1 + r)^i} + \frac{F}{(1+r)^T}.
\end{align}
The agreeable price of a coupon bond is thus given by
\begin{align}
    B_0 = 
    \sum_{i = 1}^T \frac{C}{(1 + r)^i} + \frac{F}{(1+r)^T} =
    \frac{\xi_T}{(1 + r)^T}.
\end{align}
Where the pay-off $\xi_T$ at time $T$ is given by
\begin{align}
    \xi_T := \sum_{i = 1}^T C(1 + r)^{T -i} + F,
\end{align}
in other words the fair price of the coupon bond (as well as the zero-coupon bond) can be written as the discounted price of the total pay-off.
\newpage

\section{Portfolio Allocation and Risk Measures}
\textit{Try to explain what a portfolio is and how we can create (allocate) portfolio based on risk measures. In your discussion about the latter, remember to include the differences/benefits/drawbacks of considering the standard deviation, VaR and CVaR as risk measures.}

\newpage

\section{The Multi-Step Binomial Model}
\textit{Describe the dynamics of the model. Try to discuss, among other things, the way that the market information is described through time, its absence of arbitrage, and how to find a risk-neutral measure in such model.}

\newpage

\section{The First Fundamental Theorem of Asset Pricing}
\textit{Focus on explaining the hypothesis in the theorem (martingale measure, admissible strategies, arbitrage, etc.) and its applications to the pricing of financial derivatives.}

\newpage

\section{Pricing in the Binomial Model}
\textit{Explain what is the typical methodology for pricing derivatives and how such a methodology works in the Binomial model. Remember to include an example (e.g. the price of call-option).}

\newpage

\section{The Second Fundamental Theorem of Asset Pricing}
\textit{Focus on explaining the hypothesis in the theorem (martingale measure, admissible strategies, arbitrage, completeness, etc.) and its consequences on the pricing of financial derivatives. You can, in particular, show that the Binomial Model is complete while the Trinomial Model is not.}

\end{document}